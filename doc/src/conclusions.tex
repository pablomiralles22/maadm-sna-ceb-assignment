\section{Conclusiones}
En esta práctica se ha estudiado la detección de comunidades en redes sociales,
en este caso en un grafo de artículos de Amazon. En particular:

\begin{itemize}
  \item Se ha resuelto el problema mediante algoritmos voraces, como el
    algoritmo de Leiden y de Louvain para diferentes parámetros de resolución.
  \item Se ha observado el problema de resolución de la métrica de modularidad,
    así como la imposibilidad de resolver el problema independientemente del
    uso de hiperparámetros optimizados cuando las comunidades no están balanceadas
    y requieren resoluciones distintas.
  \item Se ha aplicado un algoritmo genético para la resolución del problema de
    detección de comunidades. En particular, el algoritmo multi-objetivo
    \emph{NSGA-II} con dos métricas contrapuestas seleccionadas. Se han
    estudiado las soluciones del frente de Pareto obtenido, con resultados
    buenos pero no completamente parecidos a las comunidades reales.
  \item De forma trasversal, se han puesto en práctica técnicas de visualización
    de datos apropiadas para grafos y comunidades, intentado una transmisión
    clara de los resultados obtenidos.
\end{itemize}

En general, ha sido una práctica bastante enriquecedora en cuanto a aprendizaje
de técnicas de visualización y de algoritmos y métricas de detección de
omunidades, y de optimización multi-objetivo.



