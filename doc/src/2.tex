\section{Ejercicio 2}\label{sec:ej2}

Pasamos entonces a la resolución con algoritmos genéticos muti-objetivo. Para
ello, implemento el algortimo NSGA-II, según se describe en el artículo de
\citeauthor{deb2002fastelitist} \cite{deb2002fastelitist}. Respecto al diseño,
se toman las siguientes decisiones:

\begin{description}
  \item[Representación.] Se usa la representación \emph{locus adyacency}, como
    obligaba la práctica.
  \item[Funciones objetivo.] Basado en los resultados de
    \citeauthor{shi2014Comparisonselection} \cite{shi2014Comparisonselection},
    buscamos dos funciones objetivo que estén correlacionadas negativamente,
    de manera que sean objetivos contrapuestos: mejorar uno empeora el otro.
    El problema se convierte entonces en encontrar un compromiso entre ambos.
    Uso las siguientes funciones objetivo:
    \begin{itemize}
      \item \textbf{Densidad interna media de las comunidades}. Busca que los nodos de
        la misma comunidad estén muy conectados entre sí. Si una comunidad tiene
        \(n\) nodos y \(m\) aristas, la densidad interna es \(2m/(n(n-1))\), ya que
        el número máximo de aristas posible es \(n(n-1)/2\). Por lo tanto, la métrica
        se calcula como:
        $$
        \frac{1}{|C|} \sum_{S \in C} \frac{m_s}{(n_s \cdot (n_s-1)) / 2}
        .$$

      \item \textbf{\emph{Average out degree fraction.}} Dada una comunidad \(S\),
        definimos \(I_S = \{ (u,v): u,v\in S \}\) como el conjunto de aristas internas
        a \(S\), y \(T_S = \{ (u,v): u\in S \text{ o } v \in S\}\) como el conjunto
        de aristas que tienen al menos un extremo en \(S\). La fracción de aristas
        que apuntan al exterior de \(S\) es entonces \(1 - |I_S| / |T_S|\). La métrica
        se calcula entonces como:
        $$
        \frac{1}{|C|} \sum_{S \in C} \left(1 - \frac {|I_S|} {|T_S|} \right)
        ,$$
        donde en este caso buscamos minimizar.
    \end{itemize}

    Estas dos métricas es claro que se contraponen. Cuantos más nodos añadimos a una
    comunidad, más difícil es que todos estén bien conectados entre si, y por lo tanto
    la densidad interna disminuye. Por otro lado, cuantos más nodos tenga una
    comunidad, menos potenciales aristas hacia fuera de dicha comunidad tendrá. En el
    caso extremo, una sola comunidad nos da \(0\) de \emph{average out degree fraction},
    mientras que si tenemos \(n\) comunidades, cada una con un solo nodo, cada comunidad
    tendría densidad interna máxima.

    En el artículo citado estas dos métricas aparecen correlacionadas positivamente,
    porque una se minimiza y la otra se maximiza. En este caso, mi algoritmo solo maximiza,
    de forma que trabajaré con \(1 - AVG\_ODF\) en lugar de con \(AVG\_ODF\).

  \item[Operadores de selección.] Usaré selección por torneo, minimizando el rango (número
    del frente de Pareto al ordenar por dominancia e ir extrayendo los frentes) y maximizando
    en segunda instancia la distancia de Crowding, como se indica en \cite{deb2002fastelitist}.

  \item[Operadores de cruce.] Usaré un cruce por máscara aleatoria, donde los elementos
    a \lstinline{True} serán intercambiados y los otros no. Usar operados de cruce que
    atiendan a la posición de los genes creo que no tiene sentido, ya que en un grafo el
    orden de los nodos no importa, y las operaciones deberían ser invariantes a la
    permutación de los nodos.

  \item[Operadores de mutación.] Usaré una combinación de tres mutaciones distintas. Cada
    vez que se muta a un individuo, se elige una al azar. Son las siguientes:
    \begin{itemize}
      \item \lstinline{random(ratio)}. Cada gen se muta con probabilidad
        \lstinline{ratio} por otro vecino aleatorio del nodo.

      \item \lstinline{join(ratio)}. Consideramos ahora solo los nodos cuyo gen
        es él mismo o apunta a otro gen que le apunta de vuelta. En estos
        casos, el nodo podría adherir su comunidad a otra apuntado a otro
        vecino. Para cada uno de estos, se muta con probabilidad
        \lstinline{ratio} por un vecino aleatorio del nodo.

      \item \lstinline{separate(ratio)}. Para cada nodo, con probabilidad
        \lstinline{ratio} se muta por él mismo, rompiendo entonces un enlace
        y posiblemente separando la comunidad en dos.
    \end{itemize}

    Al introducir estas tres mutaciones con parámetros, permite dar más o menos
    peso a cada una, e integrarlas en la medida que mejor resultados obtenga.
    Al poder optimizar estos hiperparámetros, el algoritmo podrá adaptar el
    nivel de exploración en zonas de comunidades grandes o pequeñas. Estos
    hiperparámetros se seleccionarán evaluando los frentes de Pareto con
    distintas métricas, mediante la librería de Python \lstinline{optuna}.
\end{description}

Los parámetros optimizados con \lstinline{optuna} se muestran en la
Tabla~\ref{tab:ej2-params}, junto con las posibilidades consideradas y los
valores seleccionados finalmente. Para su selección se consideran las métricas
de \emph{hipervolumen} y \emph{dispersión de Zitzler} vistas en clase. El
algoritmo devuelve un nuevo frente de Pareto con estas métricas, y selecciono
una solución intermedia de compromiso arbitraria. Se realizan en total \(120\)
pruebas de hiperparámetros, con \(5000\) llamadas a la función de
\emph{fitness} y un solo experimento por cada una. Serían necesarios varios
experimentos para cada combinación de hiperparámetros, obteniendo una media de
las métricas, pero esto es demasiado costoso computacionalmente.

\begin{table}[!htbp]
\centering
\begin{tabular}{|r||cccc|}
\hline
Parámetro                       & Tipo    & Rango         & Paso & Valor seleccionado \\\hline
Tamaño de población             & Entero  & $[25, 150]$ & $25$   & $50$                 \\
Probabilidad de cruce           & Decimal & $[0.5, 1]$  & $0.05$ & $0.55$               \\
Probabilidad de mutación        & Decimal & $[0, 0.5]$  & $0.05$ & $0.1$                \\
Ratio de mutación aleatoria     & Decimal & $[0, 0.2]$  & -    & $0.0047$             \\
Ratio de mutación de unión      & Decimal & $[0, 1]$    & -    & $0.86$               \\
Ratio de mutación de separación & Decimal & $[0, 0.2]$  & -    & $0.012$              \\
T (candidatos por torneo)       & Entero  & $[2, 16]$   & -    & $4$                  \\\hline
\end{tabular}
\caption{Hiperparámetros optimizados y seleccionados para el algoritmo NSGA-II.}
\label{tab:ej2-params}
\end{table}


